\ifpdf
    \graphicspath{{Chapter8/Figs/Raster/}{Chapter8/Figs/PDF/}{Chapter8/Figs/}}
\else
    \graphicspath{{Chapter8/Figs/Vector/}{Chapter8/Figs/}}
\fi

\chapter{Conclusion}
%- - Chapter arguments:
%- - - Chap 4 (Dirt): Abuse detection is broken because it operates with trying to classify uncertain boundaries
%- - - Chap 5 (LIWC): If the boundary is porous then maybe we're looking at the wrong problem, maybe the right problem is what people mean
%- - - Chap 6 (MTL): If we must look at dirt, then maybe we need to look more broadly to understand how people are trying to communicate
%- - - Chap 7 (Disembodied): But maybe this is all wrong if what we want to do is protect people but we're actually harming them.
%- - - Chap 8 (Conclusion): Well, maybe we need to rethink this all, but some components of rethinking it are in thinking about the things around the abuse, not the abuse itself.

In this dissertation I have sought to explore the content moderation infrastructures that are built for classifying textual abuse in online spaces.
The contributions of the thesis are structured around four central themes:
How the notions of ``healthy'' and ``toxic'' content are operationalised, the implications of such operationalisation and how these come to embody hegemonic imaginaries on respectability;
how large vocabulary reductions, that represent the mental and emotional states of speakers rather than their words, influence the ability of models to classify in-domain and out-of-domain data;
how different, seemingly related, tasks can be used to jointly optimise model representations to gain models that more closely come to reflect the contexts a given speaker is operating in when speaking;
and finally, how the subjective embodiments of data subjects and modellers alike are embodied in the machine learning pipeline and how these collectively come to privilege hegemonic discourses.
These four distinct themes are connected through two over-arching questions: How does the human fit into abusive language technologies and how can machine learning models for content moderation come to more closely respect and represent their humanity?

To adequately answer these questions, I address each theme in turn through a multi-disciplinary perspective that affords insights into the technical, social, and political dimensions of content moderation infrastructures for abusive texts.
The contributions in this dissertation are thus in part of theoretical nature and in part of experimental nature.
By examining the questions through theoretical and experimental lenses, I begin to uncover the political and technical complexities of content moderation infrastructures and obtain insights that are opaque when these questions are addressed purely theoretically or purely experimentally.
The dissertation has been structured such that I start and finish with primarily theoretical contributions while the primarily experimental contributions constitute the middle of the thesis.
I choose this structure to remain faithful to the machine learning pipeline for content moderation, addressing first definitional questions and then questions of modelling.
Finally, I take a step back and reflect on the machine learning pipeline from start to an end.

I begin with \cref{chap:filter}, where I consider how the concept of sanitisation is invoked and operationalised in content moderation settings through terminology such as ``healthy conversations'' \citep{Twitter:Health:2018} and ``toxic'' content \citep{Wulczyn:2017,Perspective:Github}.
Drawing on Mary Douglas' \citeyearpar{Douglas:1966} theories of social pollution and Josh Lepawsky's \citeyearpar{Lepawsky:2019} on dirt an toxicity, I argue that content moderation infrastructures can be understood as practices of meaning making and community building.
Moreover, I reposition the role of content moderation from negative removal to positive re-ordering of environments.
Such positive re-ordering seeks to maintain communities and their cultural values, e.g. the sanitisation of online platforms to constitute a child-friendly environment.
Such sanitisation of online spaces carries with it social and cultural codes of ``dirty'' and ``sanitised'', the ``toxic'' and the ``healthy''.
Precisely as \citet{Douglas:1966} reminds us ``no single item is dirty apart from a particular system of classification in which it does not fit’'' \citep[pp. vii]{Douglas}, which in terms of content moderation systems means that they are situated in specific cultural understandings of ``toxicy'' and ``healthy''.
For the moderation of language, these cultural codes are often applied to speech produced by marginalised communities.
For instance, as \citet{Davidson:2019} show, automated content moderation systems disproportionately identify African American English as being abusive while \citet{Dias:2020} show that commercial systems find white supremacist speech to be more acceptable than the speech of queer drag queens.
Thus content moderation systems engage in toxic slippage, where they on one hand fail to protect marginalised communities and on the other hand disproportionately police those communities.

Through an analysis of two different publicly available content moderation systems, I argue that the marginalising effect of content moderation systems are embodied in the modelling pipeline in three distinct processes each of which are often made opaque.
First, in the operationalisation of ``toxic'' or ``abusive'' as these definitions and understandings are derived from cultural understandings of toxicity.
Second, they are embodied in the choice of annotators and the frequent use of majority voting on each document for annotation, encoding a hegemonic perspective on acceptability.
Third, beyond the annotation procedures, dominant discourse perspectives on respectability are embedded into the modelling e.g. in the use of pre-trained embeddings that are trained on texts available on the web.
Through these three means, content moderation systems are constituted by social hegemonies while simultaneously constituting these hegemonies by the replication of them.
Thus, content moderation systems partake in culture wars on what ``acceptability'' is to constitute.
When content moderation systems engage in such conflicts in power dynamics, they must also address the ongoing reconfigurations of dirt as the cultural contexts surrounding these change.

Using Mary Douglas' framework on social pollution and dirty, I conclude that the notion of a ``sanitised'' space also raises the question of whom it is sanitised for.
Until content moderation infrastructures are centred around the experiences of marginalised communities, they will continue to systematically disenfranchise such communities from the full ability to create boundaries that can constitute their subjective experiences.

% TODO Finish the findings here once the results
\zw{Not done}
Addressing the issues raised in \cref{chap:filter} requires a reconfiguration of the entire pipeline for automated content moderation.
In \cref{chap:liwc}, I start to examine one method for such reconfiguration: vocabulary reduction and document representations.
Here, I experiment with three different document representations: a representation where the documents undergo a word-based tokenisation process, a representation where I use sub-words to minimise out-of-vocabulary items, and finally a representation where I represent each word in a sentence as its corresponding categories in the Linguistic Inquiry and Word Count \citep[LIWC,]{Pennebaker:2001} dictionary, which allows for an approximation into the mental and emotional states of the speaker.

I find two primary implications of using LIWC represented documents for modelling, the performance of the models notwithstanding.
First, the LIWC dictionary is a dictionary with a small vocabulary resulting in large reductions in the vocabulary sizes for the models.
\zw{Double check these results and add them to the chapter}
Second, as a consequence of the smaller vocabulary sizes, models trained on this representation diverge from the word-token models in the time it takes to train them, in general requiring to epochs for LIWC-based models to converge, though model processes these iterations faster, yielding faster model training in general.









In \cref{chap:mtl}, I further investigate how optimising a model for multiple tasks simultaneously can influence model performances on specific forms of abuse.
Using hard parameter sharing Multi-Task Learning \citep{Caruana:1993}, I develop Multi-Layered Perceptron models that are optimised for different forms of abuse and investigated the impact of different auxiliary tasks for each type of abuse.
I narrowed down on the number of abusive tasks from \cref{chap:liwc}, which considered five different abusive language detection tasks to three tasks in \cref{chap:mtl} that I treat as main tasks.
Moreover, in \cref{chap:mtl}, I do not collapse the classes into binary classification, as I do in \cref{chap:liwc}.
The three tasks that I consider are: Disambiguating offensive content from hateful content and non-offensive content, disambiguating between toxic and non-toxic content, and classifying racism, sexism, and the absence of these.
For auxiliary tasks, I explore two different kinds of auxiliary task.
First, I explore four auxiliary tasks for abuse detection: hate speech detection, expert annotated hate speech detection, toxicity detection, and offensive language detection.
Three of these also serve as main tasks and a dataset is not used as an auxiliary task when it is also the main task selected for optimisation.
For the auxiliary tasks that are not abusive in nature, I explore the impact of sarcasm detection, detecting moral sentiment, and detecting whether an argument is based in facts or feelings on the main task.
I compare these MLP models with three different baselines, a linear SVM, a single-task MLP, and an ensemble classifier.
I find that although not all MTL MLP models outperform all baseline models, they consistently outperform the single-task MLP when given a single auxiliary task, regardless of whether the auxiliary task is abusive in nature or not.
Moreover, I find that not all auxiliary tasks are equally suitable, for instance, I find that the \textit{Hate Expert} auxiliary task is only useful when the main task is the \textit{Hate Speech} task. 
These two datasets are annotated from the same data sample and follow the same annotation guidelines.
More generally, I find that there are benefits in main task classification performance when the auxiliary task is abuse detection and at least one of two conditions are met:
1) the main task and auxiliary task datasets are sampled from the same, or 2) the main task and auxiliary tasks share annotation goals.
The first condition arises from observing the patterns in auxiliary task impact across all three main tasks as the \textit{Offence} auxiliary task yields improvements when the main task is either the \textit{Hate Speech} task or the \textit{Toxicity} task, where data source and dataset goals match, respectively.
When the auxiliary is not abusive in nature, \textit{Sarcasm} and \textit{Moral Sentiment} auxiliary tasks stably increases performance across the different main tasks.
The \textit{Argument Basis} auxiliary task on the other hand does not provide improvements across all datasets.
When considering combinations of auxiliary tasks, I find that using exclusively abusive tasks, or exclusively non-abusive tasks tend to obtain good scores, though these tend to be outperformed when the auxiliary task setting uses a combination of abusive and non-abusive tasks.
In such settings where abusive and non-abusive tasks are combined, the best performing settings tend to be where one abusive auxiliary task is used and two or more non-abusive tasks.
Additionally, \textit{Sarcasm} frequents in most of the best performing configurations, along the different metrics.
Through multi-task learning, it is thus possible to represent different facets of how speakers act in different situations allowing for models that more closely come to embody the context of the speaker.

In \cref{chap:disembodied}, I take a step back from the experimental modelling, observing how different subjectivities are embedded into the machine learning pipeline.
I argue here that through processes of disembodiment, machine learning as a practice hides and obscures its own positionality and the subjective decisions that are embedded in the systems.
I consider four different aspects of the modelling pipeline through which subjective experiences are embedded: in the data as it is removed from the subjective bodies that create them, in the adjudication of data for supervised classification tasks by annotators who often do not exist in the data, in the person(s) designing the experiment and modelling pipeline, and in the model trained on the disembodied data.
Extending the argument put forth by \citet{Haraway:1988} to machine learning, disembodied machine learning offers a veil of ``objectivity'' behind which a hegemonic and oppressive subjective experience is given space to be constituted in and through machine learning models.
Subsequently, other experiences, often marginalised and are left as residual experiences that must either conform to or accept the experience embodied by the modelling processes and models.
Thus, the goal of ``bias-free'' machine learning becomes a fantasy that at best performs some harm reduction and at worst obscures how machine learning models and designers of such models are complicit in perpetuating harms of oppressive structures.
To ascertain the degree to which a model takes into account the subjective experiences of the people it is developed for, I propose a spectrum to consider models through.
This spectrum ranges from fully ``localised'' models, that is models that seek to be situated within the context of the datum it is derived from to ``globalised'' models, that is models that do not seek to be grounded within such context.
As modelling processes for NLP tasks, and abuse, increasingly rely on technologies such as pre-trained embeddings, pre-trained language models, and multi-task learning, the field is moving towards models that are increasingly localised in nature, most models are unlikely to be at the extreme ends of the modelling spectrum.
Past work in abuse detection that has relied on linear models, and as is the case in for most single-task models in this dissertation, however are situated as fully globalised models.
Particularly when developing models for social prediction, such as abuse detection, it becomes vital to develop models that take into account the positionality of users.
A lack of active recognition in the modelling pipeline leads to privileging hegemonic perspectives over those of marginalised people, that is privileging those who stand to be least harms of models that fit poorly to their subjective experiences over those who stand to experience the most harm from such poor fits.
I conclude the chapter by calling for an increased recognition of the people behind the machine, that is the designers of the pipelines, the human adjudicators on the data, and the data subjects whose data is being used to train models.
Only by recognising the subjective embodiments of the participating people is it possible to describe, interrogate, and address how meaning is made in each step of the modelling process and answer whether the model that has been produced truly serves the people it has been produced for.
Until such steps are taken, machine learning modelling will continue to reproduce hegemonic views, which for content moderation and abuse detection, among many other tasks, comes to mean white supremacist imaginaries on acceptability.


\vspace{5mm}
Returning to the guiding questions of how the human fits into abusive language technologies and how machine learning models for moderating online abuse expressed through text, the findings in this dissertation suggest that at the current stage, people are not well represented by the models that the field have produced and is on track to produce.
Moreover, content moderation technologies for online abuse expressed through text at present have largely not sought to model to closely represent the subjective experiences of users.
Particularly, they have failed to represent the perspectives and experiences of those who stand to be harmed most by content moderation technologies, instead focusing on goals such as having models that take a global perspective on abuse.
These issues are the result of a computing culture that seeks to abstract away subjectivity in search of, if not ``objective truths'', global consensus on inherently subjective questions.
However, as I show in this dissertation, there is vast, and largely unexplored, space for developing models that more closely seeks to represent the people and thus better make space for peoples subjective experiences.
For instance, in \cref{chap:liwc}, we see how using modelling that seeks to encode the mental and emotional state of the author yields ... % TODO Finish sentence
Moreover, as observed in \cref{chap:mtl} the use of Multi-Task Learning can allow for models to learn representations of related auxiliary tasks, such as whether a comment is sarcastic or the expressed moral sentiments, can allow for deeper engagements with the intentions of the speaker, thus moving modelling to more closely represent the speakers and their intentions.
In this dissertation, I have sought to identify challenges and opportunities for developing models for the content moderation of abuse, to create scaffolding for a path forward that centres the subjective experiences and humanity of users of content moderation technologies, bringing back into the heart of the task those that social hierarchies make most vulnerable to abuse.
