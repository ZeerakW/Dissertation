% **************************** Define Graphics Path **************************
\ifpdf
    \graphicspath{{Chapter5/Figs/Raster/}{Chapter5/Figs/PDF/}{Chapter5/Figs/}}
\else
    \graphicspath{{Chapter5/Figs/Vector/}{Chapter5/Figs/}}
\fi

\chapter{LIWC/text transformation chapter}\label{chap:liwc}

\zw{Describe problem field}
One of the key issues in machine learning for content moderation is that such systems both in deployed settings (see \autoref{chap:filter}) and in research (see \autoref{chap:intro} and \autoref{chap:nlp}) over-fit to individual tokens that see over-representation in the positive and negative classes respectively. While research efforts have been made to address such issues \cite{CITE: cite papers that try to address overfitting}, the problem of over-fitting to words and identity markers remain an open question for the field. While some such approaches have addressed this problem by replacing certain words and phrases with more general tokens \cite{CITE: Replacing token papers} or masking \cite{CITE: Masking token paper} tokens. Other work has attempted to address the problem by treating it as a problem of dataset bias \cite{CITE: Bias papers}. Here we instead we propose a different approach which serves multiple purposes of 1) minimising the vocabulary to avoid over-fitting to distributional skews in vocabulary across classes; 2) representing documents in terms of how they represent thoughts, feelings, personality, and motivations \cite{LIWC:2015}; and 3) returning modelling of hate speech and different forms of abuse to simpler dictionary-lookup methods that simplify modelling of abuse to how words are used, rather than focusing on their surface forms, to achieve similar performance as compared to more complex models.

Through the use of the Linguistic Inquiry and Word Count (LIWC) dictionary, we transform documents from large vocabularies, that are riddled with typos, spelling mistakes, and obfuscations to the functions how these words are used. As seen in \autoref{fig:liwc_transform}, the transformation of words to LIWC categories both contain deeper information about how words are used, and encode a loss of information in words that do not exist in the LIWC dictionary. However, by radically reducing the sizes of vocabularies, from several thousand unique tokens to around 1000 unique tokens that provide deeper insights than simply selecting the most frequent tokens. Moreover we show that the strength of the classification performances within the dataset will translate to other datasets, as these transformed tokens are more general than their surface form.

Focusing our attentions on simple deep neural networks and shallow models, we show that by radically reducing the vocabulary size, while increasing the contextual information of the remaining vocabulary, we can achieve strong classification performances within datasets and more importantly, on out of domain datasets.

% \zw{Describe rationale behind the methods chosen}
\zw{Provide example of transformation}
\section{Previous work}

\zw{Talk about LIWC}
\zw{Talk in depth about the references from \autoref{chap:nlp}.}

\subsection{Datasets}
Beyond the different specific models and types of input layer, the datasets themselves also differ strongly from one another across a number of attributes: the various sizes of the datasets, the platforms the datasets have been sampled from, the annotator selection, the annotation guidelines, and the aims of the datasets. As we seek to build generalisable models and compare the performance across datasets, we reduce the classification task in each dataset to a binary `abuse/not-abuse' class, as each dataset addresses different aspects and notions of abuse.
We train our models on either the Twitter dataset collected by \citet{Davidson:2017} or the Wikipedia Talk pages dataset sampled by \citet{Wulczyn:2017} as our training datasets, in part due to the sizes of the datasets and in part due to the different breadth of communicative styles. To estimate cross-dataset performance, we evaluate our models on \citet{Waseem-Hovy:2016}, \citet{Waseem:2016}, and \citet{Garcia:2019}.

\zw{INSERT: Table of BPE and LIWC vocabularies}
\zw{INSERT: Short paragraph on vocabularies.}

\subsubsection{Training Datasets}
Our first dataset for training is the dataset developed by \citet{Davidson:2017} consists of $24,784$ tweets that are sampled from Twitter using keywords obtained from \citet{Hatebase}. The dataset is annotated for ``hate speech'', ``offensive language'' and ``neither''. The collection rationale was that not all content that immediately appears to be abusive is necessarily that, and that hate speech models must be able to distinguish between what is offensive and what is hateful \cite{Davidson:2017} (please see \autoref{chap:nlp} and \autoref{chap:filter} for more in-depth discussions on the implications of label categories, their overlap and differences). The dataset was annotated by crowd-workers on FigureEight\footnote{Previously known as CrowdFlower}. Unlike most datasets for abuse, this dataset consists primarily of positive instances, with $77$\% of the (binarised) dataset belonging to the positive class.

Our second dataset used for training is the dataset presented by \citet{Wulczyn:2017}. This dataset consists of more then $100,000$ comments from Wikipedia talk pages that have been annotated for personal attacks and toxicity \cite{Wulczyn:2017}. The rationale of this dataset is that personal attacks are harmful to ongoing conversations, and that through the identification and removals of comments that poison, or toxify online conversations, more space will be left for healthy and constructive discussions (please see \autoref{chap:filter} for an in-depth consideration of the politics of what constitutes ``toxic'' and ``healthy''). The binarised distribution of documents tagged for toxicity aligns better with prior research, with the positive class consuming $\approx 9$\% of the dataset.

\subsubsection{Evaluation Datasets}
For our evaluation datasets, we use \citet{Waseem-Hovy:2016}, a dataset of $16,000$ documents that are sampled from Twitter and annotated for ``racism'', ``sexism'', and ``neither''. The dataset was annotated by two coders, who labelled $31$\% of the dataset containing as either ``sexist'' or ``racist'' content. This dataset was developed for an early exploration into automated content moderation of online hate speech. We also use the dataset by \citet{Waseem:2016} that followed this first exploration. Here the annotation guidelines remain the same while the label-set is expanded to include the intersection of racism and sexism, the ``both'' category. This dataset contains $6000$ documents, labelled by intersectional feminist activists, and another label-set annotated by crowd-workers from FigureEight. We choose the intersectional feminist tagged annotations, as \citet{Waseem:2016} show that simple computational models perform better using this tagset. The binarised positive labels occupy $15.19$\% of the dataset. Finally, we use the dataset on white-supremacist speech developed by \citet{Garcia:2019}. This dataset, unlike the previous two evaluation datasets does not stem from Twitter, but instead the data is collected from StormFront\footnote{www.stormfront.net}, a web forum dedicated to the preservation and dissemination of white supremacist ideology. This dataset contains \zw{INSERT: Number of total document counts when using the entire dataset instead of the balanced one.} documents labelled for being hateful or not hateful, with \zw{INSERT: Percentage of positive class docs} in the positive class.

\subsubsection{Dataset and Platform Affordances}

As the datasets differ quite significantly in the sizes of the raw number of documents as well as the vocabulary sizes. Moreover, as the datasets are selected from different websites with different communities, purposes, and means of interaction; the data sampled from each platform may differ in content as well as style. Considering for instance \citet{Waseem:2016}, this dataset was collected on Twitter while the maximum length of a tweet was $140$ characters. Documents are thus short as they are given an upper limit on the number of characters. On the other hand, \citet{Wulczyn:2017} collected their data from the Wikipedia Editor Talk pages, where comments are not limited by length. Additionally, these two domains differ in that conversations on Twitter may have no particular topic, conversations on Wikipedia Talk pages always refer back to a specific topic and the conversation of how to address a particular edit to a page. Finally, given Wikipedia's ongoing issues with recruiting editors from a diverse set of backgrounds \cite{CITE: Wikipedia editors issue} and Twitter's comparatively broad user base \cite{CITE: Twitter userbase by demographic ref} may influence which dialects are represented on the platforms, which patterns of speech (e.g. sociolects, slang, and shorthand) occur, and the style of the discussions and conversations.

\subsubsection{Annotator Selection}

There are some interesting discrepancies in the selection of annotators for the datasets that we apply our models to. \citet{Waseem:2016} select their annotators based on socio-political positions, controlling for a specific interpretation of abuse. On the other hand \citet{Wulczyn:2017} select their annotators from the users of the Wikipedia Talk pages. However, the Wikipedia editor community has been accused of being a highly male space that is unwelcoming to women \cite{CITE: Cite article talking about anti-women culture on wikipedia}. This suggests that the influence of their selection of annotators, who are culturally situated in the norms and culture of the Wikipedia editor community, are also likely to be less attuned to content that may be offensive to women, but is accepted communicative practices within the Wikipedia editor community.

On the other hand \citet{Garcia:2019} and \citet{Davidson:2017} select annotators that are removed from the context of the documents they are annotating. This suggests that global understandings of what constitutes abuse are possible, and that it is possible to annotate without a deep understanding of the issues and communicative practices of the specific communities that are being investigated.

While we accept that the influence of annotation guidelines have strong influences on the subject that is being examined (e.g. \citet{Davidson:2017} examine the differences in what is merely offensive and what is hateful, \citet{Garcia:2019} examine what is hateful from a white supremacist community and what is not, and \citet{Wulczyn:2017} examine things that make conversations toxic and hostile), we assume that these different guidelines and questions highlight different aspects of abuse. Through our efforts to develop methods that can identify different forms of abuse across different datasets, we accept the assumption that there are some global understandings of abuse that can be learned by machine learning models. We revisit this assumption in \autoref{chap:disembodied}.

\section{Modelling}

In order to identify the optimal hyper-parameter settings for each model, we define and search a hyper-parameter space for each model. Though some previous work \cite{Waseem:2018, CITE: Other papers that restrict vocabulary sizes} limit the vocabulary that is used to train models, we make no such limitations on the surface level tokens. Instead, for all models that use surface level representations, we pre-process the documents using the 200 dimensional Byte-Pair Encoding \cite{Heinzerling:2018} for two reasons: 1) computing the sub-words allows for a minimisation of the number of out-of-vocabulary tokens and 2) computing the sub-words also minimises the sizes of the vocabularies for each dataset. For all models that take documents represented through LIWC, we dramatically reduce the vocabulary to only the tokens that exist within LIWC, setting all other tokens to a token representing that it is out-of-vocabulary.

\subsection{Neural Models}\ref{sec:liwc_neural}

We implement and train four different model architectures with two variations each, resulting in a total of 8 different neural models for each input representation. We choose to train a Multi-Layered Perceptron, a Recurrent Neural Network, a Long-Short Term Memory network, and a Convolutional Neural Network. We choose these four models as they have each been used in previous work \cite{CITE: Find papers with Neural approaches for each of the models}. Each of the four models are trained with either a linear input layer and an embedding input layer. We choose to make this distinction as our datasets are too small to meaningfully learn embedding layers, yet for the LIWC transformed documents, pre-trained embeddings have little value, as all tokens would be out-of-vocabulary. Thus to compare comparable entities, we train a model with each type of input layer. This difference in the input layers however also implies a difference in how each document is represented: for the models with linear input layers, each document is encoded as a onehot tensor whereas the models that use an embedding layer as it's input layer the documents are represented as index-encoded tensors (please see \autoref{fig:onehot_embedding} for an illustration). For all neural models, we use gradient clipping to prevent the issue of exploding gradients \cite{Bengio:1994}. All neural models are implemented using PyTorch \cite{CITE: Pytorch paper}.

\begin{figure}
  \centering
  \includegraphics[scale=0.75]{onehot_embedding.jpg}
  \caption{Onehot and Index encoded tensors.}
  \label{fig:onehot_embedding}
\end{figure}

In order to focus on the utility of the transformed document representations, we use bare bones models with simple architectures. To address the issue of the models over-fitting to the data either by identifying spurious correlations or by over-training the model, we subject each model to dropout and early stopping criteria (please see \autoref{sec:dropoutearly} for more details on the functionality of early stopping and dropout).

We perform Bayesian Hyper Parameter Tuning using the Optuna library \cite{Optuna:2019} to identify the optimal parameters for each model and the variants of each model, evaluated using macro-F1 score. We also investigate the influence of batch sizes and learning rate. For more in depth explanation of how each model works, please refer to \autoref{sec:model_background}.\vspace{5mm}

\begin{landscape}
\begin{table}[]
\centering
\begin{tabular}{lllllllll|llllllll}
                    & \multicolumn{8}{c|}{Davidson}                       & \multicolumn{8}{c}{Wulczyn}                        \\
                    & \multicolumn{4}{c}{BPE} & \multicolumn{4}{c|}{LIWC} & \multicolumn{4}{c}{BPE} & \multicolumn{4}{c}{LIWC} \\
                    & MLP  & CNN & RNN & LSTM & MLP  & CNN  & RNN  & LSTM & MLP  & CNN & RNN & LSTM & MLP  & CNN  & RNN & LSTM \\ \hline
Shared Dimension    &      &     &     &      &      &      &      &      &      &     &     &      &      &      &     &      \\
Embedding Dimension &      &     &     &      &      &      &      &      &      &     &     &      &      &      &     &      \\
Hidden Dimension    &      &     &     &      &      &      &      &      &      &     &     &      &      &      &     &      \\
Window Size         &      &     &     &      &      &      &      &      &      &     &     &      &      &      &     &      \\
Batch Size          &      &     &     &      &      &      &      &      &      &     &     &      &      &      &     &      \\
Learning Rate       &      &     &     &      &      &      &      &      &      &     &     &      &      &      &     &      \\
Dropout             &      &     &     &      &      &      &      &      &      &     &     &      &      &      &     &      \\
Activation Function &      &     &     &      &      &      &      &      &      &     &     &      &      &      &     &      \\
Validation F1-score &      &     &     &      &      &      &      &      &      &     &     &      &      &      &     &     
\end{tabular}
\caption{Best hyper-parameter setting for each model and dataset.}
\label{tab:redux_hyperparam_search}
\end{table}
\end{landscape}

\subsubsection{Multi-Layered Perceptron}

The first neural model that we use, is a Multi-Layered Perceptron. We choose the model as it is a simple neural network, that can act as a minimal setting of the usefulness of neural networks. Our Multi-Layered Perceptron consists of either a linear input layer or an embedding input layer. The obtained representation of a batch of documents are then passed on to a linear hidden layer and passed on to a linear output layer, which is subject to a softmax layer computing probability estimates for each class. Following the first and second layer of the model architecture, we subject the output of the layer to a non-linear activation function and a dropout layer. The model architecture is depicted in \autoref{fig:liwc_mlp}. For the Multi-Layered Perceptron models, we search over the following values:

\begin{itemize}
  \item Batch size: $[16, 32, 64]$,
  \item learning rate: $[0.1, 0.01, 0.001]$
  \item dropout probability: $[0.0, 0.1, 0.2]$,
  \item hidden layer dimension: $[100, 300]$, and
  \item the activation function: $[tanh, relu]$
\end{itemize}

For the models that use an embedding layer as their input layers, we additionally search for the dimension of the embedding layer, allowing the model to search between $[100, 300]$. For the linear input layer model, the hidden dimension search functionally replaces the search over the embedding size.

\begin{figure}
  \centering
  \includegraphics[scale=0.75]{mlp.jpg}
  \caption{Multi-Layered Perceptron model architecture.}
  \label{fig:liwc_mlp}
\end{figure}


\subsubsection{Recurrent Neural Network}

The second neural model we implement a Recurrent Neural Network; we choose this model as it offers improvements over Multi-Layered Perceptron due to the introduction of the recurrence over the tokens in the documents (see \autoref{chap:nlp} for more detail). Our Recurrent Neural Network consists of an input layer, which can be a linear layer or an embedding layer, a recurrent neural network layer, a linear output layer, a dropout layer, and a softmax layer to compute the probabilities of each class. The recurrent neural network layer is provided an activation function, which is applied within the layer. 

The model is trained by first passing batches of index or onehot encoded documents through the input layer, and are passed on to the recurrent neural network layer.\footnote{We use the PyTorch implementation of the Recurrent Neural Network layer.} The resulting representation is then subject to a dropout layer before it subject to a linear layer that maps to the number of output classes. Finally, the softmax layer computes the probability estimates for each class. See \autoref{fig:liwc_rnn} for a depiction of the models. We set the activation function for the recurrent neural network to $\tanh$.

For the Recurrent Neural Networks, we perform a hyper-parameter tuning over the following parameters and values:

\begin{itemize}
  \item Batch size: $[16, 32, 64]$,
  \item learning rate: $[0.1, 0.01, 0.001]$
  \item dropout probability: $[0.0, 0.1, 0.2]$,
  \item embedding layer dimension: $[100, 300]$, and
  \item hidden layer dimension: $[100, 300]$.
\end{itemize}

\begin{figure}
  \centering
  \includegraphics[scale=0.75]{rnn.jpg}
  \caption{Recurrent Neural Network model architecture.}
  \label{fig:liwc_rnn}
\end{figure}

\subsubsection{Long-Short Term Memory}

The Long-Short Term Memory network that we implement, consists of an input layer, that similarly to the RNN and MLP can be either a linear layer or an embedding layer; a one-directional Long-Short Term Memory network layer;\footnote{We use the PyTorch implementation of the Long-Short Term Memory Network layer.} an output layer; a dropout layer; and a softmax layer to compute the probabilities. The implementation of the Long-Short Term Memory layer is such that it always uses \textit{tanh} as its non-linear activation function. We use Long-Short Term Memory networks due to their prior successes in other works \cite{CITE: LSTM papers} and because they present a development over RNNs, in that they identify information to ``forget'' in to address the issue of long-range dependencies that occur (please see \autoref{chap:nlp} for more detail).

The model is trained by passing batches of documents through the input layer prior to feeding them into the Long-Short Term Memory network layer. The output of the Long-Short Term Memory network layer is then subject to the dropout layer, before the output layer maps down to the number of label classes. Finally, the softmax layer is used to obtain an estimation of the probability distributions for each class (please see \autoref{fig:liwc_lstm} for depiction of model architecture.).

For these models, our hyper-parameter tuning considers the following parameters and values:

\begin{itemize}
  \item Batch size: $[16, 32, 64]$,
  \item learning rate: $[0.1, 0.01, 0.001]$
  \item dropout probability: $[0.0, 0.1, 0.2]$,
  \item embedding layer dimension: $[100, 300]$, and
  \item hidden layer dimension: $[100, 300]$.
\end{itemize}

\begin{figure}
  \centering
  \includegraphics[scale=0.75]{lstm.jpg}
  \caption{Long-Short Term Memory Network model architecture.}
  \label{fig:liwc_lstm}
\end{figure}

\subsubsection{Convolutional Neural Network}

For our final neural model type, we use a Convolutional Neural Network. We select this model as it has been applied previously in academic research \cite{CITE: CNN papers} and in industry (e.g. the Perspective API\footnote{https://github.com/conversationai/perspectiveapi}). Similarly to the previous model types, the input layer of the Convolutional Neural Network models can either be an embedding layer or a linear layer. The second layer of the model is a two-dimensional convolutional layer. Finally, there is an output layer and a softmax layer (See \autoref{fig:liwc_cnn} for depiction of model architecture).

For these models, we only consider the activation function, embedding, and hidden dimension in our hyper-parameter tuning, in addition to batch size and learning rate.

\begin{itemize}
  \item Batch size: $[16, 32, 64]$,
  \item learning rate: $[0.1, 0.01, 0.001]$
  \item embedding layer dimension: $[100, 300]$, and
  \item hidden layer dimension: $[100, 300]$.
  \item the activation function: $[tanh, relu]$
\end{itemize}

\begin{figure}
  \centering
  \includegraphics[scale=0.75]{cnn.jpg}
  \caption{Convolutional Neural Network model architecture.}
  \label{fig:liwc_cnn}
\end{figure}

\subsection{Baseline Models}

We develop several different baseline methods to compare our method with. For each shallow baseline model (i.e. Logistic Regression and Support Vector Machines), we train two different types: a surface-token based model that uses the surface forms of the documents (e.g. words), and a LIWC based model. For each of these models, we represent each document for training and classification as a bag-of-words after removing stop words. In addition to the aforementioned models, we also train deep neural networks that similarly rely on surface forms. Specifically, we use the models described in \autoref{sec:liwc_neural} providing surface level tokens as the input to the models.

\subsubsection{Baseline hyper-parameters}

Similarly to our neural models, we perform a parameter search to identify the optimal parameters for training our linear baseline models. For the Support Vector Machine models, we explore a regularisation strength of $C \in \{0.1, 0.02 \ldots 1.0\}$ and $penalty \in \{L1, L2\}$. For the Logistic Regression models, we explore the same values of $C$ and append \texttt{elasticnet} to the possible space of penalties, yielding $penalty \in \{L1, L2, elasticnet\}$. We report the optimal settings in \autoref{tab:liwc_baseline_linear_params}.

\begin{table}[]
\centering
\resizebox{\textwidth}{!}{%
\begin{tabular}{l|cccc|cccc}
                      & \multicolumn{4}{c|}{BPE}                                                             & \multicolumn{4}{c}{LIWC}                                                             \\ \hline
                      & \multicolumn{2}{c}{Logistic Regression} & \multicolumn{2}{c}{Support Vector Machine} & \multicolumn{2}{c}{Logistic Regression} & \multicolumn{2}{c}{Support Vector Machine} \\ \hline
                      & C               & Penalty               & C                 & Penalty                & C               & Penalty               & C                 & Penalty                \\ \hline
\cite{Davidson:2017}  & $0.8$           & L2                    & $0.2$             & L2                     & $1.0$           & L2                    & $0.8$             & L2                     \\
\cite{Wulczyn:2017}   & $1.0$           & L2                    & $0.2$             & L2                     & $0.9$           & L2                    & $1.0$             & L2
\end{tabular}%
}
\caption{Optimal parameters for linear Support Vector Machine baselines.}
\label{tab:liwc_baseline_linear_params}
\end{table}

Considering the performances on the in-domain during training, we see in \autoref{tab:redux_linear_baselines_dev} reasonable baseline performances on the validation set. The validation scores described in \autoref{tab:redux_linear_baselines_dev} are not as strong as the state-of-the-art in-domain models \cite{Salminen:2020}, in fact they are comparable to the scores reported on the test set of the in the original paper \cite{Davidson:2017} which provided initial baseline scores.\footnote{We do not report these baseline scores as their work does not identify which weighting of their F1-score was used.}. While several previous work augment the textual data with syntactic knowledge \cite{Davidson:2017} or advanced token representations \cite{Salminen:2020} to boost classification performance, we only use the byte-pair encoded documents and the LIWC encoded documents, to ensure comparability with our experimental models. Moreover, as our primary concern is learning classifiers whose performance generalise to other datasets, unlike much prior work which concerns itself with learning classifiers that perform well within the dataset, we do not take further steps towards boosting our baseline classifiers in-domain performances.

\begin{landscape}
\begin{table}[]
\centering
\begin{tabular}{lclllll}
Training Dataset          & \multicolumn{1}{l}{Document Representation} & Model                   & F1-macro & Accuracy & Precision & Recall \\
\multirow{4}{*}{Davidson} & \multirow{2}{*}{BPE}                        & Logistic Regression     & 69.57    & 89.46    & 71.43     & 69.20  \\
                          &                                             & Support Vector Machine  & 70.78    & 89.54    & 72.74     & 70.03  \\
                          & \multirow{2}{*}{LIWC}                       & Logistic Regression     &          &          &           &        \\
                          &                                             & Support Vector Machine  &          &          &           &        \\
\multirow{4}{*}{Wulczyn}  & \multirow{2}{*}{BPE}                        & Logistic Regression     & 86.47    & 95.70    & 90.13     & 83.56  \\
                          &                                             & Support Vector Machine  &          &          &           &        \\
                          & \multirow{2}{*}{LIWC}                       & Logistic Regression     &          &          &           &        \\
                          &                                             & Support Vector Machine  &          &          &           &
\end{tabular}
\caption{In-domain scores on validation set by linear baselines.}
\label{tab:redux_linear_baselines_dev}
\end{table}
\end{landscape}

\section{Experimental Models}

To evaluate which training dataset allows for better generalisation, we train our four models described in \autoref{sec:liwc_neural} and their variations on each of our training dataset, resulting in $16$ different trained models. We show the best performing model parameters on the respective validation sets in \autoref{tab:redux_hyperparam_search}. In order to gain confidence intervals, we select a subset of these models and train them with $5$ different initial random seeds, to allow us to make claims of statistical significance of our models.

We train all of our neural network models following the same training procedure. We iterate over the training dataset in multiple epochs, shuffling the order of the data at the beginning of each epoch. As we train on a single task, the loss that is propagated through the network using backpropagation is computed on the validation set for the given task. To avoid over-training our model, we implement set our models to stop training after $15$ epochs of worse, that is strictly higher, loss values. As our training procedure closes, we apply the model on each test set, allowing us to evaluate its in-domain performance as well as its out-of-domain performance.

\begin{landscape}
\begin{table}[]
\centering
\begin{tabular}{lclllll}
Training Dataset          & \multicolumn{1}{l}{Document Representation} & Model                   & F1-macro & Accuracy & Precision & Recall \\
\multirow{4}{*}{Davidson} & \multirow{2}{*}{BPE}                        & Logistic Regression     & 69.57    & 89.46    & 71.43     & 69.20  \\
                          &                                             & Support Vector Machine  & 70.78    & 89.54    & 72.74     & 70.03  \\
                          & \multirow{2}{*}{LIWC}                       & Logistic Regression     &          &          &           &        \\
                          &                                             & Support Vector Machine  &          &          &           &        \\
\multirow{4}{*}{Wulczyn}  & \multirow{2}{*}{BPE}                        & Logistic Regression     & 86.47    & 95.70    & 90.13     & 83.56  \\
                          &                                             & Support Vector Machine  &          &          &           &        \\
                          & \multirow{2}{*}{LIWC}                       & Logistic Regression     &          &          &           &        \\
                          &                                             & Support Vector Machine  &          &          &           &
\end{tabular}%
\caption{In-domain scores on validation set by linear baselines.}
\label{tab:redux_linear_baselines_dev}
\end{table}
\end{landscape}

\zw{CODING: retrain the best parameters for the models with 4 new random seeds}

\begin{table}[]
\centering
\begin{tabular}{lllll|llll}
                    & \multicolumn{4}{c|}{Onehot input layer} & \multicolumn{4}{c}{Embedding input layer}\\
                    & MLP     & RNN     & LSTM     & CNN     & MLP      & RNN      & LSTM      & CNN     \\\hline
Hidden dimension    &         &         &          &         &          &          &           &         \\
Embedding dimension &         &         &          &         &          &          &           &         \\
Activation function &         &         &          &         &          &          &           &         \\
Batch size          &         &         &          &         &          &          &           &         \\
Learning rate       &         &         &          &         &          &          &           &         \\
# Epochs            &         &         &          &         &          &          &           &         \\
Dropout             &         &         &          &         &          &          &           &
\end{tabular}
\caption{Best parameter setting for each experimental model type trained on \citet{Davidson:2017}.}
\label{tab:exp_model_parameters_davidson}
\end{table}

\begin{table}[]
\centering
\begin{tabular}{lllll|llll}
                    & \multicolumn{4}{c|}{Onehot input layer} & \multicolumn{4}{c}{Embedding input layer}\\
                    & MLP     & RNN     & LSTM     & CNN     & MLP      & RNN      & LSTM      & CNN     \\\hline
Hidden dimension    &         &         &          &         &          &          &           &         \\
Embedding dimension &         &         &          &         &          &          &           &         \\
Activation function &         &         &          &         &          &          &           &         \\
Batch size          &         &         &          &         &          &          &           &         \\
Learning rate       &         &         &          &         &          &          &           &         \\
# Epochs            &         &         &          &         &          &          &           &         \\
Dropout             &         &         &          &         &          &          &           &
\end{tabular}
\caption{Best parameter setting for each experimental model type trained on \citet{Wulczyn:2017}.}
\label{tab:exp_model_parameters_wulczyn}
\end{table}

\zw{Add results and start by rough analysis of just numbers}
\zw{Look at predictions in detail, try to identify where they still fail}
\zw{Do logistic regression and find out if there are things that the models might overfit on}

\section{Results}

\begin{landscape}
\begin{table}[]
\centering
\begin{tabular}{ccl|llll|llll}
\multicolumn{1}{l}{}                 & \multicolumn{1}{l}{}                        &                                 & \multicolumn{4}{c|}{Davidson}            & \multicolumn{4}{c}{Wulczyn}              & \multicolumn{4}{c}{Waseem}               & \multicolumn{4}{c}{Waseem-Hovy}          & \multicolumn{4}{c}{Garcia}              \\
\multicolumn{1}{l}{Training Dataset} & \multicolumn{1}{l}{Document Representation} & Model                           & F1-macro & Accuracy & Precision & Recall & F1-macro & Accuracy & Precision & Recall & F1-macro & Accuracy & Precision & Recall & F1-macro & Accuracy & Precision & Recall & F1-macro & Accuracy & Precision & Recall \\ \hline
\multirow{12}{*}{Davidson}           & \multirow{6}{*}{BPE}                        & \textit{Support Vector Machine} &          &          &           &        &          &          &           &        &          &          &           &        &          &          &           &        &          &          &           &        \\
                                     &                                             & \textit{Logistic Regression}    &          &          &           &        &          &          &           &        &          &          &           &        &          &          &           &        &          &          &           &        \\
                                     &                                             & Multi-Layered Perceptron        &          &          &           &        &          &          &           &        &          &          &           &        &          &          &           &        &          &          &           &        \\
                                     &                                             & Convolutional Neural Network    &          &          &           &        &          &          &           &        &          &          &           &        &          &          &           &        &          &          &           &        \\
                                     &                                             & Recurrent Neural Network        &          &          &           &        &          &          &           &        &          &          &           &        &          &          &           &        &          &          &           &        \\
                                     &                                             & Long-Short Term Memory network  &          &          &           &        &          &          &           &        &          &          &           &        &          &          &           &        &          &          &           &        \\
                                     & \multirow{6}{*}{LIWC}                       & \textit{Support Vector Machine} &          &          &           &        &          &          &           &        &          &          &           &        &          &          &           &        &          &          &           &        \\
                                     &                                             & \textit{Logistic Regression}    &          &          &           &        &          &          &           &        &          &          &           &        &          &          &           &        &          &          &           &        \\
                                     &                                             & Multi-Layered Perceptron        &          &          &           &        &          &          &           &        &          &          &           &        &          &          &           &        &          &          &           &        \\
                                     &                                             & Convolutional Neural Network    &          &          &           &        &          &          &           &        &          &          &           &        &          &          &           &        &          &          &           &        \\
                                     &                                             & Recurrent Neural Network        &          &          &           &        &          &          &           &        &          &          &           &        &          &          &           &        &          &          &           &        \\
                                     &                                             & Long-Short Term Memory network  &          &          &           &        &          &          &           &        &          &          &           &        &          &          &           &        &          &          &           &        
\end{tabular}
\caption{Performance of models trained on \citet{Davidson:2017} across both in-domain and out-of-domain datasets (\textit{italic} denotes baseline models).}
\label{tab:redux_performance_davidson}
\end{table}
\end{landscape}

\begin{landscape}
\begin{table}[]
\centering
\begin{tabular}{ccl|llll|llll}
\multicolumn{1}{l}{}                 & \multicolumn{1}{l}{}                        &                                 & \multicolumn{4}{c|}{Davidson}            & \multicolumn{4}{c}{Wulczyn}              & \multicolumn{4}{c}{Waseem}               & \multicolumn{4}{c}{Waseem-Hovy}          & \multicolumn{4}{c}{Garcia}              \\
\multicolumn{1}{l}{Training Dataset} & \multicolumn{1}{l}{Document Representation} & Model                           & F1-macro & Accuracy & Precision & Recall & F1-macro & Accuracy & Precision & Recall & F1-macro & Accuracy & Precision & Recall & F1-macro & Accuracy & Precision & Recall & F1-macro & Accuracy & Precision & Recall \\ \hline
\multirow{12}{*}{Wulczyn}            & \multirow{6}{*}{BPE}                        & \textit{Support Vector Machine} &          &          &           &        &          &          &           &        &          &          &           &        &          &          &           &        &          &          &           &        \\
                                     &                                             & \textit{Logistic Regression}    &          &          &           &        &          &          &           &        &          &          &           &        &          &          &           &        &          &          &           &        \\
                                     &                                             & Multi-Layered Perceptron        &          &          &           &        &          &          &           &        &          &          &           &        &          &          &           &        &          &          &           &        \\
                                     &                                             & Convolutional Neural Network    &          &          &           &        &          &          &           &        &          &          &           &        &          &          &           &        &          &          &           &        \\
                                     &                                             & Recurrent Neural Network        &          &          &           &        &          &          &           &        &          &          &           &        &          &          &           &        &          &          &           &        \\
                                     &                                             & Long-Short Term Memory network  &          &          &           &        &          &          &           &        &          &          &           &        &          &          &           &        &          &          &           &        \\
                                     & \multirow{6}{*}{LIWC}                       & \textit{Support Vector Machine} &          &          &           &        &          &          &           &        &          &          &           &        &          &          &           &        &          &          &           &        \\
                                     &                                             & \textit{Logistic Regression}    &          &          &           &        &          &          &           &        &          &          &           &        &          &          &           &        &          &          &           &        \\
                                     &                                             & Multi-Layered Perceptron        &          &          &           &        &          &          &           &        &          &          &           &        &          &          &           &        &          &          &           &        \\
                                     &                                             & Convolutional Neural Network    &          &          &           &        &          &          &           &        &          &          &           &        &          &          &           &        &          &          &           &        \\
                                     &                                             & Recurrent Neural Network        &          &          &           &        &          &          &           &        &          &          &           &        &          &          &           &        &          &          &           &        \\
                                     &                                             & Long-Short Term Memory network  &          &          &           &        &          &          &           &        &          &          &           &        &          &          &           &        &          &          &           &        
\end{tabular}
\caption{Performance of models trained on \citet{Davidson:2017} across both in-domain and out-of-domain datasets (\textit{italic} denotes baseline models).}
\label{tab:redux_performance_davidson}
\end{table}
\end{landscape}
\zw{Add results and start by rough analysis of just numbers}
\zw{Add plots for development of loss over each epoch}
\zw{Add plots for F1 score during the evaluation set}
\zw{Look at predictions in detail, try to identify where they still fail}
\zw{Do logistic regression and to find out what clear patterns there are}

While the onehot and index encoded tensors should functionally be equal to one another, we see a direct influence of the input layers on the classification scores; with all models showing stronger performance using linear input layers. We propose that the reason for such discrepancies lie in the simpler training procedure of linear layers which don't seek to find relationships between different tokens but instead simply provide a linear function, and embedding layers that seek to identify the relationships between each all tokens in the dataset.

\section{Conclusions and future work}

\zw{Some concluding remarks}
While functionally this limits the vocabulary, there is also loss of information. Future work, could then employ both simple and complex mappings of different forms of words to single tokens that cohere with the LIWC dictionary, thus limiting information loss while retaining the predictive power.

\zw{}
