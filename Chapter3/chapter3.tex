% !TEX root = ../thesis.tex
\chapter{Computational Background}\label{chap:nlp}

\ifpdf
    \graphicspath{{Chapter3/Figs/Raster/}{Chapter3/Figs/PDF/}{Chapter3/Figs/}}
\else
    \graphicspath{{Chapter3/Figs/Vector/}{Chapter3/Figs/}}
\fi
In this chapter, we will introduce related work in Natural Language Processing. Additionally, we will introduce the methods which we will employ in this thesis.

\zw{Add more new papers and more papers that are relevant to the exact narrative of this thesis}

\section{Natural Language Processing}
In the field of natural language processing (NLP), there has been a recent increase in work focused on abusive language and bias detection. Much of the research is still early-stage work, leaving much room for further inquiry, in particular on considering how biases are employed in written language. In this section, we will provide a detailed overview of recent work and trends in the topics.

\subsection{Abusive Language Detection}
% \zw{Update this}
Abusive language detection is a growing field of inquiry. Much off the early work focused on cyber-bullying \citep{Chen:2012,Cho:2013,Reynolds:2011} and profanity \citep{Sood:profanity:2012,Sood:2013} with little focus on demographically specified abuse, such as racism, sexism, and antisemitism \citep{Warner:2012}. More recently, work on demographically specified has surfaced as an independent task \citep{Waseem:2016,Waseem-Hovy:2016,Davidson:2017,Tulkens:2015,Agarwal:2016,Silva:2016,Park:2017,Samghabadi:2017}.

A large part of the previous work on hate speech detection has primarily touched upon surface level analysis of abusive language, leaving much room for work to be done. A large effort has been expended in attempting to define annotation schemes. \cite{Waseem-Hovy:2016} proposed guidelines derived from gender studies \citep{McIntosh:1988} where a document is labeled as hate speech, if it fails any point in the guidelines. 

\cite{Waseem-Hovy:2016} released a data set for sexist and racist speech on social media which is annotated using their guidelines. In their paper, they investigate the impact of several features on detecting racism and sexism. They find that characters are more discriminative for hate speech detection in line with the findings of \cite{Mehdad:2016}. In addition, \cite{Waseem-Hovy:2016} find that information about a users gender can slightly improve classification performance, however they also find that adding location information slightly harms a classifiers performance. In addition, they find that information on length negatively impacts a classifiers performance.

\cite{Ross:2016} investigate annotator agreement for anti-refugee sentiment. They instruct their annotators to follow the Twitter's guidelines for hateful content. They find that on a data set of 541 tweets, they achieve a very poor inter-annotator agreement, suggesting that it is necessary for clear and concise guidelines for annotation of abusive language.

Building on the work of \cite{Waseem-Hovy:2016} and \cite{Ross:2016}, \cite{Waseem:2016} consider the impact of annotators' knowledge of hate speech for building models for hate speech detection; they find that employing feminist annotators for labeling data sets allows for more consistent annotations and models as compared to annotators that are not screened for political opinion. \cite{Waseem:2016} consider the application of features from sarcasm detection, using Author Historical Salient Terms (AHST) proposed by \cite{Bamman:2015}. The feature is generated by computing TF-IDF scores for each user and selecting the 100 highest weighted terms. If a term then occurs both in the document being analyzed and in the AHST. \cite{Waseem:2016} find that AHST performs extremely poorly, suggesting that hate speech may generally be a one off event, rather than a continuous stream of abuse. It is our contention that another reason AHST might not work is due to the data set employed being highly imbalanced. 

\cite{Davidson:2017} seek to break down the task of hate speech detection into offensive language and hate speech and obtain labels for a Twitter data set using crowd sourced labor on CrowdFlower. 

More recently \cite{Badjatiya:2017} trained a deep convolutional neural network (CNN) on the data set annotated by \cite{Waseem-Hovy:2016}. By using a CNN on the data set \cite{Badjatiya:2017} obtain a significant improvement on Waseem and Hovy's (2016) scores improving the F1 score from  $73.89$ to $93.00$. Given the large increase in scores it is prudent to consider any potential errors. The data set \cite{Badjatiya:2017} employ, is highly imbalanced with the positive classes occupying a small minority of the labeled data, there is a risk that their model performs extremely well on the negative class but does not perform well on the positive classes. However, no error analysis is provided in the paper.

In continuation of the results obtained by \cite{Badjatiya:2017}, \cite{Park:2017} compare using a two-step logistic regression classification, and a single step CNN approach to detecting hate speech. In the single step CNN, the specific form of hate speech is directly predicted, while in the two-step classification scenario, first a classifier is trained to identify whether a document contains abuse followed by predicting the specific type of abuse it is.

A different approach is attempted by \cite{Waseem:2018}, in which they seek to combine three different datasets for abusive language detection using multi-task learning. With a hypothesis that abuse will differ between geographic and cultural locales, they seek to employ disjoint datasets and train two models, one for each data set that share parameters. We will seek to extend this work to employ more data sets of abusive language as well as related tasks, such as sentiment analysis.

Finally, \cite{Jha:2017} break ``sexism'' down into benevolent and hostile sexism. They apply the ambivalent sexism theory as proposed by \cite{Glick:1996}. The ambivalent sexism theory suggests that there are two forms of sexism, benevolent sexism, which on a surface level speaks positively on women, but on a deeper level seeks to assert their inferiority, and hostile sexism, which expresses a strictly negative point of view on women. The following examples illustrate benevolent and hostile sexism respectively: ``Women are like flowers who need to be cherished.'' and ``Jus gonna say it..again..DUMB BITCH! \#MKR''.\vspace{5mm}

As online platforms seek to remove abuse occurring on their platform, data sets that have been gathered and annotated may have the abusive documents removed, thus requiring several rounds of re-annotation of abusive language. In an attempt to deal with this, we will experiment with using documents that are assumed have a higher chance of being abusive as they are posted in forums that are known to abusive. Using these documents we will seek to build different forms of embeddings and evaluating on previously annotated data. Further, in an attempt to mitigate annotator needs, we will build an abusive language potential system which utilizes supervised methods for Named Entity Recognition (NER), Gender Identification \citep{Sap:2014}, and sentiment analysis amongst other methods. The goal of this is to identify the probability that a document has potential to contain abusive content, in efforts to exact greater control over what documents an annotator is faced with. Additionally, such a system will allow us to identify documents which are clearly abusive. Thus, we will be able to provide an automated method to create a seed set of positive documents for abusive language detection. 

\section{Multitask Learning}



\section{Summary}
In this chapter, we have introduced the NLP work related to thesis and sought to show how we will build and expand on this work.
