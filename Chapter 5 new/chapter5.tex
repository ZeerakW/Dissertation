% **************************** Define Graphics Path **************************
\ifpdf
    \graphicspath{{Chapter5/Figs/Raster/}{Chapter5/Figs/PDF/}{Chapter5/Figs/}}
\else
    \graphicspath{{Chapter5/Figs/Vector/}{Chapter5/Figs/}}
\fi

\chapter{Multitask Learning}\label{chap:mtl}

Following the considerations in \ref{chap:detection}, we see that detecting abuse can follow along a number of trajectories and 
\subsection{RQ4: Multi-Task Learning for Abusive Language Detection}\label{sub:mtl}
In this experiment we follow up on the work of \cite{Waseem:2018}, in which they seek to utilize multi-task learning for abusive language detection. In \cite{Waseem:2018} they utilize multiple abusive language data sets, in this experiment we will seek to utilize data sets for other tasks such as sentiment analysis, dependency parsing, cyber-bullying, and named entity recognition as our auxiliary tasks in order to investigate the utility of related tasks for improving abusive language detection. Our main task will be detecting abusive language. As we will use tasks where large scale data is available, we hope to improve generalisability of abusive language detection methods without the use of large scale data sets for abusive language. As such, this experiment is core to the thesis.

As there is no not a need for collecting and annotating data sets, this experiment does not rely on others.

\subsection{RQ4: Classifying Abusive Documents using Linguistic Information and Meta data}\label{sub:structuredpred}
In this experiment, we will seek to address the issue of over-fitting to the small positive classes in abusive language data sets and thus having models have confounding variables. We will aim to avoid confounding variables in our models by seeking to utilize linguistic knowledge such as the part of speech tags and dependency trees. Further, we will seek to identify whether meta data such as the creation time of the user account, the frequency of tweets, and the number of followers and followees is useful for abusive language detection. In this experiment we will be using previously published data sets \citep{Waseem:2016,Waseem-Hovy:2016,Davidson:2017} in addition to data obtained from abusive communities on Reddit. We will be applying linear models (logistic regression and SVMs) and neural network methods (LSTMs and CNNs). As this experiment seeks to improve generalisability of abusive language classifiers through the use of linguistic annotation and user meta data, it is central to the thesis.

This experiment does not explicitly require more data, however we will let this experiment depend on the data obtained through the collection of abusive communities on Reddit.


\section{Timetable}\label{sec:timetable}
In order to answer the research questions described in Section \autoref{sec:rq}, the following tasks will be carried out:
These are the experiments which we are aiming to work on throughout the remainder of this Ph.D. studentship. Items marked with an ``$^*$'' have already received ethical approval, while items marked with ``$^+$'' have yet to have an application for ethical approval granted but an application have been filed. Items marked with ``$^-$'' are covered by granted applications for ethical approval or do not require ethical approval.
\newpage

\begin{enumerate}
  \item{Socio-technical Context of Content Moderation}
    \begin{enumerate}
      \item{$^-$ Content Moderation as care}
      \item{$^-$ Legal Realities of content moderation}
      \item{$^-$ Towards a Critical NLP}
    \end{enumerate}
  \item{Fairness in NLP}
    \begin{enumerate}
      \item{$^-$ Fair Fake News Detection}
      \item{$^*$ Fair Abusive Language Detection}
    \end{enumerate}
  \item{Temporal Patterns}
    \begin{enumerate}
      \item{$^+$ About Time: Analyzing Temporal Patterns of Abuse}
      \item{$^+$ Early Predictions of Abuse}
      \item{$^+$ External Tensions}
    \end{enumerate}
  \item{Generalisability of and Improved Classification Models for Abusive Language Models}
    \begin{enumerate}
      % \item{$^-$ Exploration of (Pre-trained) Embeddings for Abusive Language Detection}
      \item{$^*$ Multi-Task Learning for Abusive Language Detection Using Out of Domain data sets}
      \item{$^-$ Classifying Abusive Documents using Linguistic Annotation and Meta data}
    \end{enumerate}
\end{enumerate}

In Table \autoref{tab:schedule}, we present a time table for the completion of each subtask.

\begin{table}[]
  \centering
  % \scriptsize
  \resizebox{\textwidth}{!}{%
    \begin{tabular}{ll|cccc|cc}
      & & \multicolumn{4}{c|}{2019} & \multicolumn{2}{c}{2020}\\\hline

      Overall Task                                                      & Subtask                                          & Q1        & Q2        & Q3        & Q4        & Q1        & Q2\\
      \textit{Literature Review}                                        &                                                  &           &           &           &           &           & \\
      \textit{RQ1: Socio-technical Context of Content Moderation}       &                                                  &           &           &           &           &           & \\
                                                                        & Content Modeation as care                        & $\bullet$ &           &           &           &           & \\
                                                                        & Legal Realities of Content Moderation            &           &           & $\bullet$ &           &           & \\
                                                                        & Towards a Critical NLP                           &           &           & $\bullet$ &           &           & \\
      \textit{RQ2: Fairness in NLP}                                     &                                                  &           &           &           &           &           & \\
                                                                        & Fair Fake News Detection                         & $\bullet$ & $\bullet$ &           &           &           & \\
                                                                        & Fair Abusive Language Detection                  &           & $\bullet$ &           & $\bullet$ &           & \\
      \textit{RQ3: Temporal Patterns}                                   &                                                  &           &           &           &           &           & \\
                                                                        & About Time: Analyzing Temporal Patterns of Abuse &           &           & $\bullet$ & $\bullet$ &           & \\
                                                                        & Early Prediction of Abusive Moments              &           &           &           &           & $\bullet$ & \\
                                                                        & External Tensions                                &           &           &           &           & $\bullet$ & \\
      \textit{RQ4: Generalizability of Abusive Language Models}         &                                                  &           &           &           &           &           & \\
                                                                        & Multi-Task Learning Using Out-Of-Domain data     &           &           & $\bullet$ &           &           & \\
                                                                        & Linguistic Annotation \& Meta data               &           &           & $\bullet$ &           &           & \\

    \end{tabular}%
    }
    
  \caption{Task Timetable}
\label{tab:schedule}
\end{table}

Table \autoref{tab:schedule} presents the timetable for the activities previously outlined. Each task is scheduled on a quarterly basis to provide an overview of the expected time necessary to complete it. For each group of tasks, a quarter is given as the period in which we will be writing the respective chapters of the thesis. At the end of each individual experiment component, we schedule a month for submission of papers to conferences and journals.

\section{Contingency Plans}

Should an experiment overrun its allotted time by a quarter or more, we will drop a subsequent experiment. For instance, should the early prediction of abusive moments take longer to implement and test than expected, then we will drop working on the experiment working on identifying external tensions and predicting external events. 
Additionally, we leave 3 months for finalizing the final chapters of the thesis. These three months also serve as a buffer for experiments and time running over without using the potential fourth year of the PhD.

\section{Summary}

In this chapter, we have broken each research question into the experiments and provide a timeline for each experiment. While there may be deviations from this timeline, we will seek to adhere to it as best possible.
